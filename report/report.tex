\documentclass{sig-alt-release2}
\usepackage{url}
\usepackage{color}
\usepackage{graphics,graphicx}

\usepackage{epsfig}
\usepackage{epstopdf}

\usepackage{colortbl}
\usepackage{multirow}
\usepackage{booktabs}
\usepackage{ifthen}  

\begin{document}
\newcommand{\todo}[1]{\textcolor{red}{#1}}
%\def\newblock{\hskip .11em plus .33em minus .07em}

%\conferenceinfo{DIM3} {2010, Glasgow, UK} 
%\CopyrightYear{2010}
%\clubpenalty=10000
%\widowpenalty = 10000

\title{{Report Name}}

%\numberofauthors{1}
%\author{
%\alignauthor
%Author Names\\
%	   \affaddr{Team D}\\
%      \affaddr{Dim3}\\
%      \affaddr{Fiona Buyers, Christopher James, Ryan Wells}\\
%             \email{\{1003648b, 1003019j, 1002253w \}@students.glasgow.ac.uk }
%}
\maketitle

\begin{abstract}
A multi-player game designed to encourage a balance between personal gain and the well being of a group.

\end{abstract}

\section{Aim of Application}

\subsection{Application Purpose}

The application itself is a multi-player game designed to encourage users to work effectively as part of a group. When users register they will be randomly assigned to one of four factions. Whilst playing the user will aim to increase not only their individual score, but the score of their faction. To do this, the user will have to buy buildings to expand their `city' and increase their population. There will be several different kinds of buildings; farms, barracks, studios and labs; each earning the user points in a different category (food, military, art and science, respectively) and all these buildings will earn money as time progresses. Points earned by the user will be added to the total for the relevant faction and the four faction totals will be displayed, showing which faction is the strongest in each area. Crisis events will occur throughout the game which will cause buildings of a certain kind to become less or more effective, for example famine or war, so in order for the user to have a successful city and gain the maximum number of points, the city they have created should be as well balanced as possible. More accomplished users will be able to unlock different personas when they have enough money, giving the user different multipliers for each of the points categories, affecting the amount of points earned for the faction. The majority of personas will not only feature positive multipliers, but also some that are detrimental to point production. This will allow users to choose what their city will focus on in terms of the types of points they wish to earn, but will also encourage users to continue playing as regular expansions of cities will be required to enable a user to afford a change in persona.

\subsection{Functional Requirements}

\begin{itemize}

\item The user's individual score should be tracked.

\item The individual score should affect the appropriate team's overall score.

\item Upon registration, the user should be randomly assigned a faction.

\item The amount of money a user currently has should be tracked and the user should not be allowed to buy buildings or personas that cost more than they have, and when an item is bought the amount of money possessed should be altered appropriately.

\item Users should earn money and points from their current buildings at set time intervals.

\end{itemize}
 
\begin{itemize}

\item	What is the purpose of the application?

\item	Eg. The application is an academic search engine called AcaSe and is it is based upon the PuppyIR Framework\cite{glassey2011framework}, which has been used to construct other such services\cite{glassey2010fifi,elliot2010fifi}. The main purpose of this web application is to provide a customized interface to services such as Google Scholar and MS Academic Search. 

\item	What are the assumptions about the aims and objectives?

\item	Describe the design goals and objectives of the application.

\item	What are the constraints of the project?

\item	Functionality List: i.e. what is the required and desired functionality?

\item	Reflective Questions: 
\item	Is the scope of the application appropriate? 
\item	Are the design goals realistic/achievable? 
\item	How complex is the application? 
\item	Is distribution across the web appropriate? 

\end{itemize}

\section{Client Interface}

the thing shows the main page of the thing which has lots of stuff, blah blah blah dribble dribble arse

\begin{figure}[!htbp]
  \caption{\textit{Main Page}}
  \begin{center}
		\includegraphics[scale=0.25]{img/w3.png}
  \end{center}
\end{figure}

the other one shows more stuff that is a lie.

\begin{figure}[!htbp]
  \caption{\textit{Game Page}}
  \begin{center}
		\includegraphics[scale=0.25]{img/w4.png}
  \end{center}
\end{figure}

The user interface has been designed to be as simple and as funcitonal as possible. 
There are three main screens: 
The index  page, provides options to login, register or view the current faction scores.
The faction scores page shows the faction that currently has the most money, and also the greatest number of points for each of the categories. This page should be accessible from within the game and without logging in, allowing unregistered players the ability to monitor the factions' scores.
The main game page shows the current number of buildings each user has, displays their current points earnings per turn and how much money they will earn each turn. From here, users will also be able to purchase buildings for their city, and personas.

\subsection{Dynamic Components}
Due to the nature of the application, frequent updating of the database is required. The majority of the updating done to the client side uses Ajax GET requests, some at a timed interval, some whenever a user performs a specific event, such as purchasing a building.

\subsection{User Interface}
The user interface was primarily designed to provide a fluid and natural experience to the user. When considering the user experience, care was taken to ensure that the interface was attractive to users but still provided enough functionality on each layer such that users could fulfil all the relevant tasks they wished to without having to change page an excessive amount of times. \\
The interface mostly followed the wireframes originally drawn up, but changes were made following feedback from the presentation earlier in the Semester. 
\begin{itemize}

\item	Draw a wireframe of the user interface 

\item	this may require several wireframes depending on the complexity of the application and the interfaces

\item	Describe the user interface.

\item	i.e. Label key input and output components: describe them.

\item	Provide a Walkthrough and explain the user interactions with application. 
\item	i.e. use cases AAAAAAAAARGH

\item	Describe the interactions associated with the dynamic components on the user interface.

\item	What calls are required to dynamically update the data on the client side?

\item	How does the user interface help the user achieve their goal, or complete their task? 

\item	Is the user interface intuitive, appealing, usable, etc?

\item	What technologies are used on the client side? 

\item	What are the reasons for your choices? i.e. what is the advantages and disadvantages of using this technology? 

\item	What other options are there? 

\end{itemize}

\section{Application Architecture}
\begin{itemize}

\item	N-Teir Architecture Diagram - chris has this

\item	i.e. data flow diagram between the interface/client, middle ware, and backend services/data repos

\item	Describe the data model i.e. what data needs to be stored or persisted by the application?

\item	What are the relationships within the data model.
\item	i.e. use ER diagram and explain. - chris I DONT KNOW, we never had one really

\item	Describe the backend services used (if any).

\item	Reflective Questions: 
\item	How have you ensured that there is a separation of concerns? 
\item	What other technology could have been used instead of django? 
\item	What are the advantages of using a Web Application Framework over other technology? 
\item	And, what are the disadvantages?
\end{itemize}

\section{Message Parsing}
\begin{itemize}

\item	On the architecture diagram, Identify and label the main messages that will be parsed through the application.
\item	or alternatively (and preferably) include sequence diagrams to denote the sequence of communications parse between clients and servers.

\item	Describe the messages that are parsed back and forth through the application.

\item	For the main transactions - describe the payload of the messages 
\item	i.e. What are the contents of the messages? i.e. include sample XML, XHTML, JSON, etc of one or two messages.

\item	What is the format of the messages? 

\item	Why this format? 

\item	What other formats could be used, what are the advantages and disadvantages of these other formats?
\end{itemize}


\section{Implementation Notes}

\begin{itemize}

\item Views - What are the main views that you have implemented and what do they do?



\item URL Mapping Schema - what is your URL mapping and schema?

\item External Services  - what external services does your application include and what handlers did you include?

\item	Functionality Checklist (which functionality is completed) - everything except crises 

\item	Known Issues (what kind of works, what kind of errors to do you get)

\item What technologies have been used and are required for the application. Include a list or table of all the technologies, standards, and protocols that will be required.
\end{itemize}

\section{Reflective Summary}
{\bf For the Implementation Report Only:}
\begin{itemize}

\item	What have you learnt through the process of development? 

\item	How did the application of frameworks help or hinder your progress? 

\item	What problems did you encounter? - timing?

\item	What were your major achievements?
\end{itemize}

\section{Summary and Future Work}
\begin{itemize}

\item	Summary of application and its current state.

\item	Include a list or table of all the technologies, standards, and protocols that will be required.

\item	What are the limitations?

\item Plans for future development - crisis events, more personas and buildings.

\end{itemize}

\section{Acknowledgements}
Our thanks to Lief and Euan for their comments, suggestions and interest they showed in the project throughout the duration of the course. And our thanks to the peer reviewers for their feedback.
Thanks also go to Tom Whitely for his evaluation of the interface and the suggestions he made.

\bibliographystyle{abbrv}
\bibliography{sig-proc}

\end{document}
